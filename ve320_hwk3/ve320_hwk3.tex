\documentclass[11pt,oneside,a4paper]{article}
\usepackage{amsmath}
\everymath{\displaystyle}
\setlength\parindent{0pt}
\begin{document}

\begin{center}Assignment3\\\end{center} 
\begin{flushright}Yuan Xiaojie\\5093709092\\\end{flushright} 

\ \\

1.\:(a) \(f(E_F)=\frac{1}{1+e^{(E_F-E_F/kT)}}=\frac{1}{2}\) \\ 

\quad (b) \(f(E_c+kT)=\frac{1}{1+e^{(E_c+kT-E_c)/kT}}=\frac{1}{1+e}=0.269\) \\

\quad (c) \(f(E_c+kT)=1-f(E_c+kT)\) \\

\hspace{8.5mm} \(\Rightarrow f(E_c+kT)=\frac{1}{2}\) \\

\hspace{8.5mm} \(\Rightarrow \frac{1}{1+e^{(E_c+kT-E_F)/kT}}=\frac{1}{2}\) \\

\hspace{8.5mm} \(\Rightarrow E_c+kT-E_F=0\) \\

\hspace{8.5mm} \(\Rightarrow E_F=E_c+kT \) \\
\\

2.\:(a) \(N_c=2\left[\frac{m_n^*kT}{2\pi\hbar^2}\right]^{3/2} = 2\left[\frac{(9.11\times10^{-31}\times1.18)(0.0259\times1.6\times10^{-19})}{(2\pi)(6.63\times10^{-34}/2\pi)^2}\right]^{3/2}10^{-6}\) \\

\hspace{15mm} \(=3.21\times10^{19}\) cm\(^{-3}\cdot\)eV \\

\quad (b) \(N_v=2\left[\frac{m_p^*kT}{2\pi\hbar^2}\right]^{3/2}=2\left[\frac{(9.11\times10^{-31}\times0.81)(0.0259\times1.6\times10^{-19})}{(2\pi)(6.63\times10^{-34}/2\pi)^2}\right]^{3/2}10^{-6}\)

\hspace{15mm} \(=1.83\times10^{19}\) cm\(^{-3}\cdot\)eV \\
\\

3.\:(a) \(n=p=n_i=10^{10}\) cm\(^{-3}\) \\

\quad (b) Since \(N_D\ll N_A,\ N_D\ll n_i\) \\

\hspace{8.5mm} \(n\approx N_D=10^{13}\) cm\(^{-3}\) \\

\hspace{8.5mm} \(p\approx n_i^2/N_D=10^7\) cm\(^{-3}\) \\

\quad (c) Since \(N_A\ll N_D,\ N_A\ll n_i\) \\

\hspace{8.5mm} \(n\approx N_D=10^{17}\) cm\(^{-3}\) \\

\hspace{8.5mm} \(p\approx n_i^2/N_D=10^3\) cm\(^{-3}\) \\

\quad (d) \(p=\frac{N_A-N_D}{2}+\left[\left(\frac{N_A-N_D}{2}\right)^2+n_i^2\right]^{1/2}\approx N_A-N_D=2\times10^{17}\) cm\(^{-3}\) \\

\hspace{8.5mm} \(n=n_i^2/p=5\times10^2\) cm\(^{-3}\) \\

\quad (e) \(E_i=\frac{E_c+E_v}{2}+\frac{3}{4}kTln\left(\frac{m_p^*}{m_n^*}\right)=\frac{E_c+E_v}{2}-0.0073\) \\

\hspace{8.5mm} \(E_i-E_F=kTlnN_A/n_i=0.0259ln(10^{17}/10^{10})=0.417\) \\

\hspace{8.5mm} Then \(E_F=E_i-0.47=\frac{E_c+E_v}{2}-0.4243\) \\

\hspace{8.5mm} Assume \(E_v=0\), which implies \(E_c=E_G\) \\

\hspace{8.5mm} \(E_F=\frac{E_G}{2}-0.4243=\frac{1.12}{2}-0.4243=0.1357\) eV \(>E_v+3kT\) \\

\hspace{8.5mm} So \(E_F\) is in the nondegenerate area. \\
\\

4.\ From \(E_i=\frac{E_c+E_v}{2}+\frac{3}{4}kTln\left(\frac{m_p^*}{m_n^*}\right)\) and \(E_G=E_c-E_v\), we can derive that \\

\hspace{8.5mm} \(E_i-E_v=\frac{E_G}{2}+\frac{3}{4}kTln\left(\frac{m_p^*}{m_n^*}\right)\) \\

\hspace{8.5mm} \(E_c-E_i=\frac{E_G}{2}+\frac{3}{4}kTln\left(\frac{m_p^*}{m_n^*}\right)\) \\

\quad (a) Si at 300K with \(m_n^*=1.182m_0\) and \(m_p^*=0.81m_0\) \\

\hspace{8.5mm} \(E_i-E_v=0.5527\) eV \\

\hspace{8.5mm} \(E_c-E_v=0.5673\) eV \\

\quad (b) GaAs at 300K with \(m_n^*=0.067m_0\) and \(m_p^*=0.524m_0\) \\

\hspace{8.5mm} \(E_i-E_v=0.67\) eV \\

\hspace{8.5mm} \(E_c-E_v=0.75\) eV \\ 
\\
\\
\\
\\

5.\:(a) \\
\\
\\
\\
\\
\\
\\
\\
\\
\\
\\
\\
\\
\\
\\

\quad (b) \(E_i\) will lie above the midgap.

\hspace{8.5mm} Because \(DOS\) is smaller in the conduction band, equal number of 

\hspace{8.5mm} states in two bands can be filled only if \(E_i\) lies close to \(E_c\). \\

\quad (c) \(E_i=\frac{E_c+E_v}{2}+\frac{3}{4}kTln\left(\frac{m_p^*}{m_n^*}\right)=\frac{E_c+E_v}{2}+0.04\) \\

\hspace{8.5mm} So \(E_i\) lies 0.04eV above the midgap. \\

\quad (d) \(N_c=2\left[\frac{m_n^*kT}{2\pi\hbar^2}\right]^{3/2}=4.26\times10^{17}\) cm\(^{-3}\) \\

\hspace{8.5mm} \(N_v=2\left[\frac{m_p^*kT}{2\pi\hbar^2}\right]^{3/2}=9.41\times10^{18}\) cm\(^{-3}\) \\

\hspace{8.5mm} For nondegenerate semiconductors \\

\hspace{8.5mm} \(E_v+3kT\leq E_F\leq E_c-3kT\) \\

\hspace{8.5mm} \(n=N_ce^{(E_F-E_c)/kT}\leq N_ce^{-3}\) \\

\hspace{8.5mm} \(p=N_ce^{(E_v-E_F)/kT}\leq N_ve^{-3}\) \\

\hspace{8.5mm} Also, for \(n\)-type GaAs, \(n\approx N_D\); for \(p\)-type GaAs, \(p\approx N_A\). Then \\

\hspace{8.5mm} \((N_D)_{max}\approx N_ce^{-3}=2.21\times10^{16}\) cm\(^{-3}\) \\

\hspace{8.5mm} \((N_A)_{max}\approx N_ve^{-3}=4.68\times10^{17}\) cm\(^{-3}\) \\

\end{document}
